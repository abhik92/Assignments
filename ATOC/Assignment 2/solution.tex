\documentclass[addpoints,12pt]{exam}
\printanswers
\usepackage{amsmath,amssymb,amsthm}
\usepackage{textcomp}

\begin{document}
\hrule
\vspace{3mm}
\noindent
{\sf CS6014: Advanced Theory of Computation  \hfill Author: Karthik Abinav - CS10B057}
\vspace{3mm}\\
\noindent
{\sf Problem Set \#2 }
\vspace{3mm}
\hrule

\begin{questions}
%Question 1
\question
\begin{parts}
%Part (a)
\part
$L$ = \{$M$ : $M$ has an useless state $q$ \} \newline
Language $L$ is undecidable. Suppose $L$ was decidable then we can use that as a sub routine to solve $\overline{HP}$ as follows:
\begin{itemize}
 \item 
 Consider a new machine $\grave{M}$ which has two states - start and accept state. It writes its input $y$ on a seperate tape.
 \item
 It then runs $M$ on input $x$ . If $M$ halts on $x$ it accepts its input $y$.
\end{itemize}

  The machine $\grave{M}$ can be described as : \newline
  \[
    \grave{M} = \left\{ 
  \begin{array}{l l}
     \grave{M}~has~no~useless~state~ & \quad \text{if $M$ halts on $x$}\\
     \grave{M}~has~an~useless~accept~state~ & \quad \text{if $M$ does not halt on $x$}\\
  \end{array} \right.
\]

Since our assumption is that $L$ is decidable implies there exists a total turing machine $K$ which accpets $L$ . Giving $\grave{M}$ 
as input to $K$ , we can totally compute if $M$ halts on $x$ . But we know there is no such total turing machine. Hence, language $L$ is not decidable.

%Part (b)
\part
$L$ = \{$(M,w)$: M moves its head left during computation of w \} \newline
This language is decidable. Suppose, during computation of w, the head moves left once, we are done nad can accept $(M,w)$ . \newline
If it deosnt not move left till it reaches the first blank symbol - Let $q_1$ be the state at which it first reads the blank symbol. Let $N$ be the number of states of machine $M$. Simulate the machine for a further $N+1$ steps. If the head moved left during this time then accept $(M,w)$ . Else reject. Because : \newline
$\delta(q_1,blank) = q_j~\forall j\in[1,N]$ . By pigeon hole principle, it should return to a previously visited state after $N+1$ steps.   

%Part (c)
\part
$L$ = \{($M1$,$M2$): $L(M1)$ = $\overline{L(M2)}$\} \newline
$L$ is undecidable. If $L$ was decidable we can use the total turing machine $K$ accepting $L$ as a sub routine to solve $HP$ as follows: \newline
Consider two machines $M1$ and $M2$ which does the following:
\begin{itemize}
 \item 
  Both machines simulate $M$ on $x$ . If $M$ halts on $x$ then machine $M1$ accepts its input $y$ , while machine $M2$ rejects input $y$.
  
\end{itemize}
  So the language of machine $M1$ can be described as : 
  
  \[
    L(M1) = \left\{ 
  \begin{array}{l l}
     \Sigma^\ast & \quad \text{if $M$ halts on $x$}\\
     \varnothing & \quad \text{if $M$ does not halt on $x$}\\
  \end{array} \right.
\]

  Similarly the language accpeted by machine $M2$ can be described as:
  \[
    L(M2) = \left\{ 
  \begin{array}{l l}
     \varnothing & \quad \text{if $M$ halts on $x$}\\
     \varnothing & \quad \text{if $M$ does not halt on $x$}\\
  \end{array} \right.
\]

  Giving machines ($M1$ , $M2$) as input to machine $K$ can help in totally computing if machine $M$ halts on input $x$. Since, there is no such total turing machine, hence no such total turing machine $K$ exists.
  
 % Part (d)
 \part
 $L$ = \{$M$ : $\exists x\in\Sigma^\ast$ s.t. $M$ runs forever on input $x$\} \newline
 This language is undecidable . If this language was decidable by a total turing machine $K$ , then we can use it as a sub routine to solve $\overline{HP}$ as follows: \newline
 Consider machine $\grave{M}$ which does the following:
 
 \begin{itemize}
  \item 
  $\grave{M}$ writes its input $y$ on a seperate track.
  \item
  It runs machine $M$ on input $x$. If $M$ halts on $x$ , then it accepts its input $y$.
  
 \end{itemize}
 
 The machine $\grave{M}$ can be described as follows:
   \[
    \grave{M} = \left\{ 
  \begin{array}{l l}
     For~all~inputs~\grave{M}~doesnt~run~forever & \quad \text{if $M$ halts on $x$}\\
     There~exists~an~input~for~which~\grave{M}~runs~forever & \quad \text{if $M$ does not halt on $x$}\\
  \end{array} \right.
\]
  Giving $\grave{M}$ as input to machine $K$ can help in totally computing if machine $M$ halts on input $x$. Since, there is no such total machine, hence no such total turing machine $K$ exists. 
 
 %Part (e)
 \part
 
 $L$ = \{$M$: $M$ accepts atleast one plaindromic string\} \newline
$L$ is undecidable. If $L$ was decidable we can use the total turing machine $K$ accepting $L$ as a sub routine to solve $HP$ as follows: \newline
Consider machine $\grave{M}$ which does the following:
\begin{itemize}
 \item 
  Machine $\grave{M}$ simulates $M$ on input $x$.
  \item
  If $M$ halts on $x$, then $\grave{M}$ accepts its input $y$
  
\end{itemize}
  So the language of machine $M$ can be described as : 
  
  \[
    L(M) = \left\{ 
  \begin{array}{l l}
     \Sigma^\ast & \quad \text{if $M$ halts on $x$}\\
     \varnothing & \quad \text{if $M$ does not halt on $x$}\\
  \end{array} \right.
\]

  Since, $\Sigma^\ast$ contains atleast one palindromic string,
  Giving machine $M$ as input to machine $K$ can help in totally computing if machine $M$ halts on input $x$. Since, there is no such total turing machine, hence no such total turing machine $K$ exists.
  
 %Part (f)
 \part
  
 $L$ = \{$M$: $M$ accepts only plaindromic strings\} \newline
$L$ is undecidable. If $L$ was decidable we can use the total turing machine $K$ accepting $L$ as a sub routine to solve $HP$ as follows: \newline
Consider machine $\grave{M}$ which does the following:
\begin{itemize}
 \item 
  Machine $\grave{M}$ simulates $M$ on input $x$.
  \item
  If $M$ halts on $x$, then $\grave{M}$ checks if its input $y$ is a palindrome. If yes, it accepts $y$ , else rejects $y$.
  
\end{itemize}
  So the language of machine $M$ can be described as : 
  
  \[
    L(M) = \left\{ 
  \begin{array}{l l}
     P & \quad \text{if $M$ halts on $x$}\\
     \varnothing & \quad \text{if $M$ does not halt on $x$}\\
  \end{array} \right.
\]
  
  Where $P$ is the set of all palindromes.\newline
  Giving machine $M$ as input to machine $K$ can help in totally computing if machine $M$ halts on input $x$. Since, there is no such total turing machine, hence no such total turing machine $K$ exists.
  
  %Part (g)
  \part
  $L$ = \{($M1$,$M2$): $L(M1)\bigcap L(M2) \neq \varnothing $ \} \newline
$L$ is undecidable. If $L$ was decidable we can use the total turing machine $K$ accepting $L$ as a sub routine to solve $HP$ as follows: \newline
Consider two machines $M1$ and $M2$ which does the following:
\begin{itemize}
 \item 
  Both machines simulate $M$ on $x$ . If $M$ halts on $x$ then both machines $M1$ and $M2$ accepts their inputs $y1$,$y2$ respectively.
  
\end{itemize}
  So the language of machine $M1$ can be described as : 
  
  \[
    L(M1) = \left\{ 
  \begin{array}{l l}
     \Sigma^\ast & \quad \text{if $M$ halts on $x$}\\
     \varnothing & \quad \text{if $M$ does not halt on $x$}\\
  \end{array} \right.
\]

  Similarly the language accpeted by machine $M2$ can be described as:
  \[
    L(M2) = \left\{ 
  \begin{array}{l l}
     \Sigma^\ast & \quad \text{if $M$ halts on $x$}\\
     \varnothing & \quad \text{if $M$ does not halt on $x$}\\
  \end{array} \right.
\]
  And the Language $L(M1)\bigcap L(M2)$ can be described as : 
  \[
   L(M1)\bigcap L(M2)  = \left\{ 
  \begin{array}{l l}
     \Sigma^\ast & \quad \text{if $M$ halts on $x$}\\
     \varnothing & \quad \text{if $M$ does not halt on $x$}\\
  \end{array} \right.
\]
  
  Giving machines ($M1$ , $M2$) as input to machine $K$ can help in totally computing if machine $M$ halts on input $x$. Since, there is no such total turing machine, hence no such total turing machine $K$ exists.
 
\end{parts}

 %Question 2
 \question
	\begin{parts}
	%Part 1
	\part
	(107) $L$ = \{$M$ : $L(M)~=~rev~L(M)$\} \newline	
		This language is undecidable. \newline
		If $L$ was decidable we can use the total turing machine $K$ accepting $L$ as a sub routine to solve $HP$ as follows: 		\newline
Consider machine $\grave{M}$ which does the following:
\begin{itemize}
 \item 
  Machine $\grave{M}$ simulates $M$ on input $x$.
  \item
  If $M$ halts on $x$, then $\grave{M}$ checks if its input $y$ is a palindrome. If yes, it accepts $y$ , else rejects $y$.
  
\end{itemize}
  So the language of machine $M$ can be described as : 
  
  \[
    L(M) = \left\{ 
  \begin{array}{l l}
     P & \quad \text{if $M$ halts on $x$}\\
     \varnothing & \quad \text{if $M$ does not halt on $x$}\\
  \end{array} \right.
\]
  
  Where $P$ is the set of all palindromes.\newline
  Giving machine $M$ as input to machine $K$ can help in totally computing if machine $M$ halts on input $x$. Since, there is no such total turing machine, hence no such total turing machine $K$ exists. 
	%Part 2
	\part	
	
	%Part 3
	\part

	%Part 4
	\part	
	
	%Part 5
	\part
	
	%Part 6
	\part
	
	
	\end{parts}	 
 %Question 3
 \question
 \begin{parts}
  % Part (a)
  \part
  $L = \{a^p~:~p~is~a~prime~number\}$ \newline 
  This language is not regular but is decidable.
  %Part (b)
  \part
  For a given turing machine $M$,\newline
  $L = \{1^n~:~M~halts~on~input~1^n\}$ \newline 
  Since, its a fixed machine $M$ , the transitions of $M$ can be hardwired in the transitions of the machine $K$ simulating it. And hence the set of tape alphabets is
  singleton. 
  %Part (c)
  \part
  $ FIN = \{M~:~L(M)~is~finite\}$,\newline 
  Generally, we represent the encoding of $M$ using the bits 0's and 1's. However, this can be viewed as the nth lexicographic permutation of 0's and 1's.
  Hence, encoding $M$ as $a^n$ , where n represents the nth lexicographic permutation of 0's and 1's. Hence every machine can be encoded over a singleton alphabet.
  
 \end{parts}

 
 %Question 4
 \question
 \begin{parts}
  %Part (a)
  \part
  $\leq_{m}$ is reflexive and transitive,but not symmetric.\newline
  \underline{Reflexive}: A $\leq_{m}$ A via the identity function $\sigma (x) = x$ \newline
  \underline{Transitive}: If A $\leq_{m}$ B via a map $\sigma$ and B $\leq_{m}$ C via the map $\tau$ , then A $\leq_{m}$ C via the map $\tau\circ\sigma$ \newline
  \underline{Symmetric}: $\leq_{m}$ is not symmetric because the function $\sigma$ is not invertible. \newline \newline
  
  $\leq_{T}$ is reflexive and transitive,but not symmetric.\newline
  \underline{Reflexive}: A $\leq_{T}$ A .Given a A as an oracle, the OTM just has to return the value of its query to the oracle to accept/reject A.   \newline
  \underline{Transitive}: If A $\leq_{T}$ B , then given B as an oracle, A is totally computable. Similarly B $\leq_{T}$ C implies given C as an oracle , B is totally computable.
                          Now, an OTM with C as an oracle, can compute A totally as follows. Using C as the oracle, compute and store intermidiate value of B. Now this value can serve
                          as an answer from an oracle and hence A can be calculated. Calculation of B doesnt cause looping (B $\leq_{T}$ C). Hence, calculation of A doesnt cause looping.
                          Hence A $\leq_{T}$ C. \newline
  \underline{Symmetric}: $\leq_{T}$ is not symmetric.Example : $FIN \leq_{T} REG $ , but $REG \nleq_{T} FIN $. \newline
  
  %Part (b)
  \part
  L is decidable $=>$ L $\leq_{m}$ $1^\ast0^\ast$ \newline
  
  To show this we have to find a $\sigma : \Sigma^\ast$ \textrightarrow $\Sigma^\ast$ :
  \begin{itemize}
   \item 
   $\sigma$ is totally computable
   \item
   $\forall x, x\in L <=> \sigma(x)\in1^\ast0^\ast$ 
  \end{itemize}
  
  Since, L is decidable , there exists a total turing machine $K$ such that, $L(K)=L$ .
  Let $\sigma$ be defined as follows :
  
   \[
    \sigma(x) = \left\{ 
  \begin{array}{l l}
     10 & \quad \text{if $K$ accepts $x$}\\
     01 & \quad \text{if $K$ rejects $x$}\\
  \end{array} \right.
  \]
  
  $\sigma$ is totally computable since $K$ is total. \newline
  $x\in L <=> K~accepts~x <=>( \sigma(x) = 10 \in 1^\ast0^\ast $) \newline
  
  Therefore, L $\leq_{m}$ $1^\ast0^\ast$ \newline
  
  L is decidable $<=$ L $\leq_{m}$ $1^\ast0^\ast$ \newline
   
    RHS implies there exists a totally computable $\sigma$ such that $\forall x, x\in L <=> \sigma(x)\in1^\ast0^\ast$ . \newline
    For any input $x$ , calculating $\sigma(x)\in1^\ast0^\ast$ , can be done by a total turing machine.(Keep 4 states. when on state 1 implies seeing 1's. First time a 0 is seen,
    move to state 2. Or if end of input is seen, move to state 3 and accept. In state 2 ,if end of input is reached go to state 3 and accept. If a 1 is seen, move to state 4 and reject).
    Hence, $x\in L$ can be totally computed . Therefore , $L$ is decidable.
   
 \end{parts}

 
 
 %Question 5
 \question
 
 
 
 \end{questions}
\end{document}


