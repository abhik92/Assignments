\documentclass[addpoints,12pt]{exam}
\printanswers
\usepackage{amsmath,amssymb,complexity,amsthm}
\usepackage{graphicx}
\usepackage{textcomp}

\newtheorem{theorem}{Theorem}
\newtheorem{corollary}[theorem]{Corollary}
\newtheorem{lemma}[theorem]{Lemma}
\newtheorem{observation}[theorem]{Observation}
\newtheorem{proposition}[theorem]{Proposition}
\newtheorem{claim}[theorem]{Claim}
\newtheorem{definition}[theorem]{Definition}

\begin{document}
\hrule
\vspace{3mm}
\noindent
{\sf CS6014: Advanced Theory of Computation  \hfill Author: Karthik Abinav - CS10B057}
\vspace{3mm}\\
\noindent
{\sf Problem Set \#2 }
\vspace{3mm}
\hrule

\begin{questions}
%Question 1
\question
\begin{parts}
%Part (a)
\part
$L$ = \{$M$ : $M$ has an useless state $q$ \} \newline
Language $L$ is undecidable. Suppose $L$ was decidable then we can use that as a sub routine to solve $\overline{HP}$ as follows:
\begin{itemize}
 \item 
 Consider a new machine $\grave{M}$ which has two states - start and accept state. It writes its input $y$ on a seperate tape.
 \item
 It then runs $M$ on input $x$ . If $M$ halts on $x$ it accepts its input $y$.
\end{itemize}

  The machine $\grave{M}$ can be described as : \newline
  \[
    \grave{M} = \left\{ 
  \begin{array}{l l}
     \grave{M}~has~no~useless~state~ & \quad \text{if $M$ halts on $x$}\\
     \grave{M}~has~an~useless~accept~state~ & \quad \text{if $M$ does not halt on $x$}\\
  \end{array} \right.
\]

Since our assumption is that $L$ is decidable implies there exists a total turing machine $K$ which accpets $L$ . Giving $\grave{M}$ 
as input to $K$ , we can totally compute if $M$ halts on $x$ . But we know there is no such total turing machine. Hence, language $L$ is not decidable.

%Part (b)
\part

%Part (c)
\part
$L$ = \{($M1$,$M2$): $L(M1)$ = $\overline{L(M2)}$\} \newline
$L$ is undecidable. If $L$ was decidable we can use the total turing machine $K$ accepting $L$ as a sub routine to solve $HP$ as follows: \newline
Consider two machines $M1$ and $M2$ which does the following:
\begin{itemize}
 \item 
  Both machines simulate $M$ on $x$ . If $M$ halts on $x$ then machine $M1$ accepts its input $y$ , while machine $M2$ rejects input $y$.
  
\end{itemize}
  So the language of machine $M1$ can be described as : 
  
  \[
    L(M1) = \left\{ 
  \begin{array}{l l}
     \Sigma^\ast & \quad \text{if $M$ halts on $x$}\\
     \varnothing & \quad \text{if $M$ does not halt on $x$}\\
  \end{array} \right.
\]

  Similarly the language accpeted by machine $M2$ can be described as:
  \[
    L(M2) = \left\{ 
  \begin{array}{l l}
     \varnothing & \quad \text{if $M$ halts on $x$}\\
     \varnothing & \quad \text{if $M$ does not halt on $x$}\\
  \end{array} \right.
\]

  Giving machines ($M1$ , $M2$) as input to machine $K$ can help in totally computing if machine $M$ halts on input $x$. Since, there is no such total turing machine, hence no such total turing machine $K$ exists.
  
 % Part (d)
 \part
 $L$ = \{$M$ : $\exists x\in\Sigma^\ast$ s.t. $M$ runs forever on input $x$\} \newline
 This language is undecidable . If this language was decidable by a total turing machine $K$ , then we can use it as a sub routine to solve $\overline{HP}$ as follows: \newline
 Consider machine $\grave{M}$ which does the following:
 
 \begin{itemize}
  \item 
  $\grave{M}$ writes its input $y$ on a seperate track.
  \item
  It runs machine $M$ on input $x$. If $M$ halts on $x$ , then it accepts its input $y$.
  
 \end{itemize}
 
 The machine $\grave{M}$ can be described as follows:
   \[
    \grave{M} = \left\{ 
  \begin{array}{l l}
     For~all~inputs~\grave{M}~doesnt~run~forever & \quad \text{if $M$ halts on $x$}\\
     There~exists~an~input~for~which~\grave{M}~runs~forever & \quad \text{if $M$ does not halt on $x$}\\
  \end{array} \right.
\]
  Giving $\grave{M}$ as input to machine $K$ can help in totally computing if machine $M$ halts on input $x$. Since, there is no such total machine, hence no such total turing machine $K$ exists. 
 
 %Part (e)
 \part
 
 $L$ = \{$M$: $M$ accepts atleast one plaindromic string\} \newline
$L$ is undecidable. If $L$ was decidable we can use the total turing machine $K$ accepting $L$ as a sub routine to solve $HP$ as follows: \newline
Consider machine $\grave{M}$ which does the following:
\begin{itemize}
 \item 
  Machine $\grave{M}$ simulates $M$ on input $x$.
  \item
  If $M$ halts on $x$, then $\grave{M}$ accepts its input $y$
  
\end{itemize}
  So the language of machine $M$ can be described as : 
  
  \[
    L(M) = \left\{ 
  \begin{array}{l l}
     \Sigma^\ast & \quad \text{if $M$ halts on $x$}\\
     \varnothing & \quad \text{if $M$ does not halt on $x$}\\
  \end{array} \right.
\]

  Since, $\Sigma^\ast$ contains atleast one palindromic string,
  Giving machine $M$ as input to machine $K$ can help in totally computing if machine $M$ halts on input $x$. Since, there is no such total turing machine, hence no such total turing machine $K$ exists.
  
 %Part (f)
 \part
  
 $L$ = \{$M$: $M$ accepts only plaindromic strings\} \newline
$L$ is undecidable. If $L$ was decidable we can use the total turing machine $K$ accepting $L$ as a sub routine to solve $HP$ as follows: \newline
Consider machine $\grave{M}$ which does the following:
\begin{itemize}
 \item 
  Machine $\grave{M}$ simulates $M$ on input $x$.
  \item
  If $M$ halts on $x$, then $\grave{M}$ checks if its input $y$ is a plaindrome. If yes, it accepts $y$ , else rejects $y$.
  
\end{itemize}
  So the language of machine $M$ can be described as : 
  
  \[
    L(M) = \left\{ 
  \begin{array}{l l}
     P & \quad \text{if $M$ halts on $x$}\\
     \varnothing & \quad \text{if $M$ does not halt on $x$}\\
  \end{array} \right.
\]
  
  Where $P$ is the set of all palindromes.\newline
  Giving machine $M$ as input to machine $K$ can help in totally computing if machine $M$ halts on input $x$. Since, there is no such total turing machine, hence no such total turing machine $K$ exists.
  
  
  %Part (g)
  \part
  $L$ = \{($M1$,$M2$): $L(M1)\bigcap L(M2) \neq \varnothing $ \} \newline
$L$ is undecidable. If $L$ was decidable we can use the total turing machine $K$ accepting $L$ as a sub routine to solve $HP$ as follows: \newline
Consider two machines $M1$ and $M2$ which does the following:
\begin{itemize}
 \item 
  Both machines simulate $M$ on $x$ . If $M$ halts on $x$ then both machines $M1$ and $M2$ accepts their inputs $y1$,$y2$ respectively.
  
\end{itemize}
  So the language of machine $M1$ can be described as : 
  
  \[
    L(M1) = \left\{ 
  \begin{array}{l l}
     \Sigma^\ast & \quad \text{if $M$ halts on $x$}\\
     \varnothing & \quad \text{if $M$ does not halt on $x$}\\
  \end{array} \right.
\]

  Similarly the language accpeted by machine $M2$ can be described as:
  \[
    L(M2) = \left\{ 
  \begin{array}{l l}
     \Sigma^\ast & \quad \text{if $M$ halts on $x$}\\
     \varnothing & \quad \text{if $M$ does not halt on $x$}\\
  \end{array} \right.
\]
  And the Language $L(M1)\bigcap L(M2)$ can be described as : 
  \[
   L(M1)\bigcap L(M2)  = \left\{ 
  \begin{array}{l l}
     \Sigma^\ast & \quad \text{if $M$ halts on $x$}\\
     \varnothing & \quad \text{if $M$ does not halt on $x$}\\
  \end{array} \right.
\]
  
  Giving machines ($M1$ , $M2$) as input to machine $K$ can help in totally computing if machine $M$ halts on input $x$. Since, there is no such total turing machine, hence no such total turing machine $K$ exists.
  
  
  
 \end{parts}

 %Question 2
 \question
 
 %Question 3
 \question
 
 %Question 4
 \question
 \begin{parts}
  %Part (a)
  \part
  
  %Part (b)
  \part
  L is decidable $=>$ L $\leq_{m}$ $1^\ast0^\ast$ \newline
  
  To show this we have to find a $\sigma : \Sigma^\ast$ \textrightarrow $\Sigma^\ast$ :
  \begin{itemize}
   \item 
   $\sigma$ is totally computable
   \item
   $\forall x, x\in L <=> \sigma(x)\in1^\ast0^\ast$ 
  \end{itemize}
  
  Since, L is decidable , there exists a total turing machine $K$ such that, $L(K)=L$ .
  Let $\sigma$ be defined as follows :
  
   \[
    \sigma(x) = \left\{ 
  \begin{array}{l l}
     10 & \quad \text{if $K$ accepts $x$}\\
     01 & \quad \text{if $K$ rejects $x$}\\
  \end{array} \right.
  \]
  
  $\sigma$ is totally computable since $K$ is total. \newline
  $x\in L <=> K~accepts~x <=>( \sigma(x) = 10 \in 1^\ast0^\ast $) \newline
  
  Therefore, L $\leq_{m}$ $1^\ast0^\ast$ \newline
  
  L is decidable $<=$ L $\leq_{m}$ $1^\ast0^\ast$ \newline
   
    RHS implies there exists a totally computable $\sigma$ such that $\forall x, x\in L <=> \sigma(x)\in1^\ast0^\ast$ . \newline
    For any input $x$ , calculating $\sigma(x)\in1^\ast0^\ast$ , can be done by a total turing machine.(Keep 4 states. when on state 1 implies seeing 1's. First time a 0 is seen,
    move to state 2. Or if end of input is seen, move to state 3 and accept. In state 2 ,if end of input is reached go to state 3 and accept. If a 1 is seen, move to state 4 and reject).
    Hence, $x\in L$ can be totally computed . Therefore , $L$ is decidable.
   
 \end{parts}

 
 
 %Question 5
 \question
 
 
 
 \end{questions}
\end{document}


