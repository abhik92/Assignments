\documentclass[addpoints,12pt]{exam}
\printanswers
\usepackage{amsmath,amssymb,complexity,amsthm}
\usepackage{graphicx}

\newtheorem{theorem}{Theorem}
\newtheorem{corollary}[theorem]{Corollary}
\newtheorem{lemma}[theorem]{Lemma}
\newtheorem{observation}[theorem]{Observation}
\newtheorem{proposition}[theorem]{Proposition}
\newtheorem{claim}[theorem]{Claim}
\newtheorem{definition}[theorem]{Definition}

\begin{document}
\hrule
\vspace{3mm}
\noindent
{\sf IITM-CS6014 : Advanced Theory of Computation  \hfill Given on: Aug 28, 8am}
\vspace{3mm}\\
\noindent
{\sf Problem Set \#2 \hfill Due on : Sep 11, 5pm}
\vspace{3mm}
\hrule
{\small
\begin{itemize}
\item The instructions remain the same as last problem set. Answers need not be written in \LaTeX. However, from Problem Set 3 onwards, this will be compulsory. No PDFs will be accepted unless it is produced from a \LaTeX file.
\end{itemize}}
\hrule

\begin{questions}
\question
Consider the following three problems. Formulate the problems as languages and
argue whether they are decidable or undecidable. (You can apply Rice's theorems wherever possible.)
\begin{parts}
\part An \emph{unused state} of a Turing machine is a state that is never
entered on any input string. The problem is to determine whether a Turing
machine has any useless states.
\part Given a Turing machine $M$ on input $x$, does the machine ever move its head left at any point during its computation on $w$.
\part Given two Turing machines, $M_1$ and $M_2$, check if they accept complimentary languages.
\part Given a Turing machine, test whether there exist a string on which the machine runs for ever.
\part Given a Turing machine, test whether it accepts at least one palindrome string.
\part Given a Turing machine, test whether it accepts only palindromes.
\part Given two Turing machines, $M_1$ and $M_2$, check if they accept a common string.
\end{parts}

\question
Kozen 104-110, 115.

\question
Over a singleton alphabet, describe:
\begin{parts}
\part a language which is not regular, but is decidable.
\part a language which is semi-decidable but not decidable.
\part a language which is in $\Sigma_2$ but not in $\Sigma_1$.
\end{parts}

\question
\begin{parts}
\part View $\le_m$ and $\le_T$ as relation defined over the set of languages.
Argue whether they are reflexive, symmetric and transitive.
\part Show that $L$ is decidable if and only if $L \le_m 1^*0^*$.
\end{parts}

\question
Show that $FIN \le_T REG$. Suppose you are given an oracle which will
always answer questions of the form, "Is $L(M)$ a regular set?" truthfully.
Show how to use such an oracle to decide questions of the form, 
"Is $L(M)$ finite?"

\end{questions}
\end{document}
