\documentclass[12pt]{exam}
\printanswers
\usepackage{amsmath,amssymb,complexity}
\begin{document}
\hrule
\vspace{3mm}
\noindent 
{\sf CS6014 : Advanced Theory of Computation  \hfill Author: Karthik Abinav(CS10B057)}
\vspace{3mm}\\
\noindent 
{\sf Problem Set \#5 }
\vspace{3mm}
\hrule

\begin{questions}

\question
Let $A$ be an $\NP$-complete language and $B$ be in $\P$. Prove that if $A \cap B = \phi$, then  $A \cup B$ is $\NP$-complete. What can you say about the complexity of $A \cup B$ if $A$ and $B$ are not known to be disjoint?

\question
Show that a {\sc DNF} formula can be converted in polynomial time to a {\sc CNF} formula, with possibly more number of variables , preserving satisfiability. Show that if $\P \ne \NP$, there cannot be a polynomial time algorithm that switches a {\sc CNF} formula to an {\sc DNF} formula preserving satisfiability.
\begin{solution}
\newline
\textbf{Part a} \newline
The formula $\phi$ in {\sc DNF} will be as $P_{1} \vee P_{2} \vee P_{3} .. $ . Every $P_{i}$ will be of the form $x_{1} \wedge x_{2} \wedge x_{3} ..$
Now , take every clause of the form $P_{i} \vee P_{j}$ . \newline
If either $P_{i}$ or $P_{j}$ is a variable, then directly distribute the expression over the $\vee$ to get an expression in {\sc CNF}. \newline
Else, introduce a new symbol $y_{i}$ and make the clauses $(y_{i} \vee P_{i}) \wedge (\overline{y_{i}} \vee P_{j})$. \newline
This will preserve the satisfiability of the $(P_{i} \vee P_{j})$ expression as follows:
\begin{itemize}
 \item If both $P_{i}$ and $P_{j}$ is false, putting any value to $y_{i}$ will preserve satisfiability.
 \item Similarly if both $P_{i}$ and $P_{j}$ is true, then putting any value to $y_{i}$ will preserve satisfiability.
 \item Suppose one of them is true and other is false. WLG let $P_{i}$ be true. Then putting $y_{i}$ as true will preserve satisfiability.
\end{itemize}

Now, since $y_{i}$ is a variable, the subclauses can now be distributed in the same way as above. This will take atmost length of $P_{i}$ time.
And total time taken to convert will be $|\text{number of clauses}| \ast |\text{Time taken to expand each clause}|$ . And this will 
take polynomial time in length of the expression.

\textbf{Part b} \newline
We can show this by proving the contrapositive of the statement. i.e. If there exists a polynomial time conversion from {\sc CNF} to {\sc DNF}, then $\P = \NP$.
\newline

Consider, any formula in the {\sc CNF}. Now, use the algorithm and convert it into a {\sc DNF}. The formula will be of the form, $P_{1} \vee P_{2} ..$
and having the same satisfiability as the {\sc CNF}. Now, scan through each of the $P_{i}$ and see if any clause has both the variables $x_{i}$ and $\overline{x_{i}}$.
If not, assign to all variables in that clause of form $x_{j}$ a true and $\overline{x_{j}}$ a false value and for the variables not in that clause any value.Then report satisfiable. Else, if there is no such clause, report unsatisfiable. \newline
Hence, $\SAT$ problem can be solved in polynomial time, since the scanning takes atmost the length of the clause. And since, $\SAT$ is $\NP$-Complete,
Every problem in $\NP$ can be decided in polynomial time. And hence $\P = \NP$.

\end{solution}


\question
Give a polynomial time algorithm for 2-$\SAT$ problem that we stated in class. Given a CNF formula $\phi$, where each clause has atmost two literals, test satisfiability. Is 2-$\SAT$ in $\NL$? Argue.

\question
Consider the complexity class $\DP$ ({\sc D} stands for \textit{difference}) as the set of problems $L$ such that $L = A \cap B$ where $A$ and $B$ are two languages in $\NP$ and $\co\NP$ respectively. Argue that $\DP \subseteq \P^{\NP}$. Argue that {\sc Exact-Clique} is $\DP$-complete.

\question
Cook-Levin Theorem is proved by reducing any $L \in \NP$ to $\SAT$. Is there any relation between the number of accepting paths of the non-deterministic machine (deciding $L$) and the number of satisfying assignments of the formula produced by the reduction? Check the same for the $\NP$-completeness proof for {\sc Independent Set Problem} that we did in class.

\end{questions}

%Add your answer like this.
%\begin{solution}
%
%\end{solution}

\end{document}
