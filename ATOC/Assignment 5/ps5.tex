\documentclass[12pt]{exam}
\printanswers
\usepackage{amsmath,amssymb,complexity}
\begin{document}
\hrule
\vspace{3mm}
\noindent 
{\sf CS6014 : Advanced Theory of Computation  \hfill Author: Karthik Abinav(CS10B057)}
\vspace{3mm}\\
\noindent 
{\sf Problem Set \#5 \hfill Collaborators:Saikrishna}
\vspace{3mm}
\hrule

\begin{questions}

\question
Let $A$ be an $\NP$-complete language and $B$ be in $\P$. Prove that if $A \cap B = \phi$, then  $A \cup B$ is $\NP$-complete. What can you say about the complexity of $A \cup B$ if $A$ and $B$ are not known to be disjoint?
\begin{solution}
 
 $A \cup B$ belongs to $\NP$. Because given an $x$ we have to check if $x \in B$ or $x \in A$. And to check if $x \in A$ takes $\NP$ time. \newline
 $A \cup B$ is $\NP$-hard \newline
 
 To show this we have to show that $\forall L \in \NP$ $L \leq_{m}^{p}$  $A \cup B$.
 i.e.
 $x \in L$ $\Leftrightarrow$ $\sigma(x) \in A \cup B$.
 If we construct such a $\sigma$ computable in polynomial time we are done. \newline
 
 Since $A$ is $\NP$ complete, there exists a $\sigma_{1}$ computable in polynomial time such that,\newline
 $x \in L$ $\Leftrightarrow$ $\sigma_{1}(x) \in A$. \newline
 
 Construct the $\sigma$ such that, $\sigma(x) = \sigma_{1}(x)$ if $\sigma_{1}(x) \notin B$.\newline
 $\sigma(x) = fixed~string~w$ such that $w \notin A$ and $w \notin B$.\newline
 
 Since, $A \cap B = \varnothing$. Hence, $\sigma(x) \in B \Leftrightarrow \sigma(x) \notin A$.
 
 Therefore, $\sigma(x)$ satisfies the reduction for $L$ to $A \cup B$.\newline

 Suppose, it is not known about $A \cap B$, then the above reduction wont work. We can just conclude that $A \cup B \in \NP$ and cant
 comment about the hardness. (Unless $A \cap B = A$ or $A \cap B = B$ , in which cases $A \cup B$ will be  $\NP$-complete and $\P$ respectively).
 
 
\end{solution}


\question
Show that a {\sc DNF} formula can be converted in polynomial time to a {\sc CNF} formula, with possibly more number of variables , preserving satisfiability. Show that if $\P \ne \NP$, there cannot be a polynomial time algorithm that switches a {\sc CNF} formula to an {\sc DNF} formula preserving satisfiability.
\begin{solution}
\newline
\textbf{Part a} \newline
The formula $\phi$ in {\sc DNF} will be as $P_{1} \vee P_{2} \vee P_{3} .. $ . Every $P_{i}$ will be of the form $x_{1} \wedge x_{2} \wedge x_{3} ..$
Now , take every clause of the form $P_{i} \vee P_{j}$ . \newline
If either $P_{i}$ or $P_{j}$ is a variable, then directly distribute the expression over the $\vee$ to get an expression in {\sc CNF}. \newline
Else, introduce a new symbol $y_{i}$ and make the clauses $(y_{i} \vee P_{i}) \wedge (\overline{y_{i}} \vee P_{j})$. \newline
This will preserve the satisfiability of the $(P_{i} \vee P_{j})$ expression as follows:
\begin{itemize}
 \item If both $P_{i}$ and $P_{j}$ is false, putting any value to $y_{i}$ will preserve satisfiability.
 \item Similarly if both $P_{i}$ and $P_{j}$ is true, then putting any value to $y_{i}$ will preserve satisfiability.
 \item Suppose one of them is true and other is false. WLG let $P_{i}$ be true. Then putting $y_{i}$ as true will preserve satisfiability.
\end{itemize}

Now, since $y_{i}$ is a variable, the subclauses can now be distributed in the same way as above. This will take atmost length of $P_{i}$ time.
And total time taken to convert will be $|\text{number of clauses}| \ast |\text{Time taken to expand each clause}|$ . And this will 
take polynomial time in length of the expression.

\textbf{Part b} \newline
We can show this by proving the contrapositive of the statement. i.e. If there exists a polynomial time conversion from {\sc CNF} to {\sc DNF}, then $\P = \NP$.
\newline

Consider, any formula in the {\sc CNF}. Now, use the algorithm and convert it into a {\sc DNF}. The formula will be of the form, $P_{1} \vee P_{2} ..$
and having the same satisfiability as the {\sc CNF}. Now, scan through each of the $P_{i}$ and see if any clause has both the variables $x_{i}$ and $\overline{x_{i}}$.
If not, assign to all variables in that clause of form $x_{j}$ a true and $\overline{x_{j}}$ a false value and for the variables not in that clause any value.Then report satisfiable. Else, if there is no such clause, report unsatisfiable. \newline
Hence, $\SAT$ problem can be solved in polynomial time, since the scanning takes atmost the length of the clause. And since, $\SAT$ is $\NP$-Complete,
Every problem in $\NP$ can be decided in polynomial time. And hence $\P = \NP$.

\end{solution}


\question
Give a polynomial time algorithm for 2-$\SAT$ problem that we stated in class. Given a CNF formula $\phi$, where each clause has atmost two literals, test satisfiability. Is 2-$\SAT$ in $\NL$? Argue.

\begin{solution}
  Consider any formula $\phi$ in the $2-\SAT$ form. \newline
  It will be of the form $(x_{1} \vee x_{2}) \wedge (x_{3} \vee x_{4})..$. \newline
  Like the previous question, the technique will be to reduce formulas of the form $(x_{1} \vee x_{2}) \wedge (\overline{x_{1}} \vee x_{3})$ to
  $(x_{2} \vee x_{3})$ . The consistency in satisfiability follows from distributing the $\vee$ over the $\wedge$ and using the claim in question 2.\newline
  
  Now, the algorithm will scan through the formula for pairs of clauses having $x_{i}$ in one and $\overline{x_{i}}$ in the other and reduce them.
  There can be atmost O($n^2$) such pairs for each variable. Next replace all clauses having $(x_{i} \vee \overline{x_{i}})$ with a 0. 
  Now , initial formula is satisfiable, if the new obtained formula is not a fallacy(i.e. 0). \newline
  
  The running time of the algorithm takes polynomial time in the number of variables times the number of clauses.
  
  
\end{solution}


\question
Consider the complexity class $\DP$ ({\sc D} stands for \textit{difference}) as the set of problems $L$ such that $L = A \cap B$ where $A$ and $B$ are two languages in $\NP$ and $\co\NP$ respectively. Argue that $\DP \subseteq \P^{\NP}$. Argue that {\sc Exact-Clique} is $\DP$-complete.

\begin{solution}
\textbf{Part a} \newline
  Let $L \in \DP$ . Implies there exists $A$ and $B$ such that $A \cap B$ = $L$ . Now construct an OTM with a $\SAT$ as an oracle.
  The description of the machine is as follows:
  \begin{itemize}
   \item Given an input $x$, checks if $x \in \A$ by doing the poly time reduction to $\SAT$ via a function $\sigma_{1}$ and queries the oracle 
   for $\sigma_{1}(x) \in \SAT$.
   \item checks if $x \notin \overline{B}$ by computing the poly time reduction to $\SAT$ via the function $\sigma_{2}$ and queries the oracle.
   If $x \notin \overline{B} \implies x \in B$ because B is $\co\NP$.
   \item The machine accepts if both queries give accept. Else, rejects.
  \end{itemize}
  Since, the reductions are poly time, this OT machine also runs in polynomial time. Hence, $L \in \DP \implies L \in \P^{\NP}$.
  Therefore, $\DP \subseteq \P^{\NP}$. \newline
\textbf{Part b} \newline  
   To check $L \in $ {\sc Exact-Clique} we have to check two parts - $(G,k) \in$ {\sc Clique} and $(G,k+1) \notin$ {\sc Clique} 
   Define ${\sc clique_{k}}$ as set of all graphs with a clique of size k. and similarly, ${\sc clique_{k+1}}$ as set of all graphs with a clique
   of size k+1.
   {\sc exact-clique} $\in \DP$ . The set $A = {\sc clique_{k}}$ and the set $B = {\sc clique_{k+1}}$. And $A$ is $\NP$ complete and $B$ is $\co\NP$ complete.
   \newline
   $\forall L \in \DP$
   There exists a $\sigma_{1}$ computable in poly time(via machine $M_{1}$) such that, $x \in L \Leftrightarrow \sigma_{1}(x) \in A$ and
   a poly time computable(via machine $M_{2}$) $\sigma_{2}$ such that , $x \in L \Leftrightarrow \sigma_{2}(x) \in B$. (Because $A$ and $B$ are $\NP$ complete and $\co\NP$ complete
   respectively). \newline
 
   Construct a machine $N$ which simulates $M_{1}$ and $M_{2}$.
   Given an input $y$, it computes $\sigma_{1}$ by simulating $M_{1}$ and checks if $\sigma_{1}(y) \in A$. \newline
   Then computes $\sigma_{2}$ by simulating $M_{2}$ ad checks if $\sigma_{2}(y) \in B$.\newline
   If both are true accepts, else rejects.\newline
   Since both these machines run in poly time the machine $N$ also runs in poly time.Also, the language accepted by $N$ is {\sc Exact-Clique} .
   This shows that {\sc Exact-clique} is $\DP$ complete.
  
  
  
\end{solution}


\question
Cook-Levin Theorem is proved by reducing any $L \in \NP$ to $\SAT$. Is there any relation between the number of accepting paths of the non-deterministic machine (deciding $L$) and the number of satisfying assignments of the formula produced by the reduction? Check the same for the $\NP$-completeness proof for {\sc Independent Set Problem} that we did in class.

\end{questions}

\begin{solution}
\newline
 In the proof we did in class, we constructed the clause in $\SAT$ by considering every bit in the configuration from start to end as a variable,
 and constructed a clause whose truth value will be 1 exactly when the sequence of configurations will be accepting one. Hence, the number of accepting
 configurations is precisely the number of satisfying assignments to the constructed $\SAT$ clause. \newline
 
 In case of the {\sc Independent Set Problem}, we choose a subset($S$) of vertices such that, \newline
 $x_{i} \in S$ if $x_{i} = 0$ in the assignment to $3-\SAT$.And all the clauses which evaluate to 1 were also added into $S$. \newline
 Hence, the number of sets $S$ is precisely the number of sets i's for which $x_{i} = 0$. This is the number of satisfiable assignments for $3-\SAT$.
 This is twice the number of satisfiable assignments for $\SAT$. Because, suppose the clause in $\SAT$ was of the form, $x_{1} \vee x_{2} \vee x_{3}$,
 then it is converted to $3-\SAT$ as $(x_{1} \vee x_{2} \vee y_{1}) \wedge (\overline{y_{1}} \vee x_{3} \vee y_{2})$. And for a fixed values of $x_{i}$ s, there
 is a choice on the last variable $y_{n}$ to be either 0 or 1(in the above example $y_{2}$). Hence, the number of independent sets is twice the
 number of accepting paths in the non deterministic machine.
\end{solution}

%Add your answer like this.
%\begin{solution}
%
%\end{solution}

\end{document}
